\documentclass[11pt,]{article}
\usepackage[left=1in,top=1in,right=1in,bottom=1in]{geometry}
\newcommand*{\authorfont}{\fontfamily{phv}\selectfont}
\usepackage[]{mathpazo}


  \usepackage[T1]{fontenc}
  \usepackage[utf8]{inputenc}




\usepackage{abstract}
\renewcommand{\abstractname}{}    % clear the title
\renewcommand{\absnamepos}{empty} % originally center

\renewenvironment{abstract}
 {{%
    \setlength{\leftmargin}{0mm}
    \setlength{\rightmargin}{\leftmargin}%
  }%
  \relax}
 {\endlist}

\makeatletter
\def\@maketitle{%
  \newpage
%  \null
%  \vskip 2em%
%  \begin{center}%
  \let \footnote \thanks
    {\fontsize{18}{20}\selectfont\raggedright  \setlength{\parindent}{0pt} \@title \par}%
}
%\fi
\makeatother




\setcounter{secnumdepth}{0}

\usepackage{color}
\usepackage{fancyvrb}
\newcommand{\VerbBar}{|}
\newcommand{\VERB}{\Verb[commandchars=\\\{\}]}
\DefineVerbatimEnvironment{Highlighting}{Verbatim}{commandchars=\\\{\}}
% Add ',fontsize=\small' for more characters per line
\usepackage{framed}
\definecolor{shadecolor}{RGB}{248,248,248}
\newenvironment{Shaded}{\begin{snugshade}}{\end{snugshade}}
\newcommand{\AlertTok}[1]{\textcolor[rgb]{0.94,0.16,0.16}{#1}}
\newcommand{\AnnotationTok}[1]{\textcolor[rgb]{0.56,0.35,0.01}{\textbf{\textit{#1}}}}
\newcommand{\AttributeTok}[1]{\textcolor[rgb]{0.77,0.63,0.00}{#1}}
\newcommand{\BaseNTok}[1]{\textcolor[rgb]{0.00,0.00,0.81}{#1}}
\newcommand{\BuiltInTok}[1]{#1}
\newcommand{\CharTok}[1]{\textcolor[rgb]{0.31,0.60,0.02}{#1}}
\newcommand{\CommentTok}[1]{\textcolor[rgb]{0.56,0.35,0.01}{\textit{#1}}}
\newcommand{\CommentVarTok}[1]{\textcolor[rgb]{0.56,0.35,0.01}{\textbf{\textit{#1}}}}
\newcommand{\ConstantTok}[1]{\textcolor[rgb]{0.00,0.00,0.00}{#1}}
\newcommand{\ControlFlowTok}[1]{\textcolor[rgb]{0.13,0.29,0.53}{\textbf{#1}}}
\newcommand{\DataTypeTok}[1]{\textcolor[rgb]{0.13,0.29,0.53}{#1}}
\newcommand{\DecValTok}[1]{\textcolor[rgb]{0.00,0.00,0.81}{#1}}
\newcommand{\DocumentationTok}[1]{\textcolor[rgb]{0.56,0.35,0.01}{\textbf{\textit{#1}}}}
\newcommand{\ErrorTok}[1]{\textcolor[rgb]{0.64,0.00,0.00}{\textbf{#1}}}
\newcommand{\ExtensionTok}[1]{#1}
\newcommand{\FloatTok}[1]{\textcolor[rgb]{0.00,0.00,0.81}{#1}}
\newcommand{\FunctionTok}[1]{\textcolor[rgb]{0.00,0.00,0.00}{#1}}
\newcommand{\ImportTok}[1]{#1}
\newcommand{\InformationTok}[1]{\textcolor[rgb]{0.56,0.35,0.01}{\textbf{\textit{#1}}}}
\newcommand{\KeywordTok}[1]{\textcolor[rgb]{0.13,0.29,0.53}{\textbf{#1}}}
\newcommand{\NormalTok}[1]{#1}
\newcommand{\OperatorTok}[1]{\textcolor[rgb]{0.81,0.36,0.00}{\textbf{#1}}}
\newcommand{\OtherTok}[1]{\textcolor[rgb]{0.56,0.35,0.01}{#1}}
\newcommand{\PreprocessorTok}[1]{\textcolor[rgb]{0.56,0.35,0.01}{\textit{#1}}}
\newcommand{\RegionMarkerTok}[1]{#1}
\newcommand{\SpecialCharTok}[1]{\textcolor[rgb]{0.00,0.00,0.00}{#1}}
\newcommand{\SpecialStringTok}[1]{\textcolor[rgb]{0.31,0.60,0.02}{#1}}
\newcommand{\StringTok}[1]{\textcolor[rgb]{0.31,0.60,0.02}{#1}}
\newcommand{\VariableTok}[1]{\textcolor[rgb]{0.00,0.00,0.00}{#1}}
\newcommand{\VerbatimStringTok}[1]{\textcolor[rgb]{0.31,0.60,0.02}{#1}}
\newcommand{\WarningTok}[1]{\textcolor[rgb]{0.56,0.35,0.01}{\textbf{\textit{#1}}}}



\title{White Advantage in Chess and How to Counter It  }
 



\author{\Large Jun Won (Lakon) Park (Group
Leader)\vspace{0.05in} \newline\normalsize\emph{79453940, Group Leader,
data collection, data analysis, report-writing}   \and \Large Sarah
Li\vspace{0.05in} \newline\normalsize\emph{60136959, data collection,
data analysis, report-writing}  }


\date{}

\usepackage{titlesec}

\titleformat*{\section}{\normalsize\bfseries}
\titleformat*{\subsection}{\normalsize\itshape}
\titleformat*{\subsubsection}{\normalsize\itshape}
\titleformat*{\paragraph}{\normalsize\itshape}
\titleformat*{\subparagraph}{\normalsize\itshape}


\usepackage{natbib}
\bibliographystyle{plainnat}
\usepackage[strings]{underscore} % protect underscores in most circumstances



\newtheorem{hypothesis}{Hypothesis}
\usepackage{setspace}


% set default figure placement to htbp
\makeatletter
\def\fps@figure{htbp}
\makeatother

\usepackage{booktabs}
\usepackage{longtable}
\usepackage{array}
\usepackage{multirow}
\usepackage{wrapfig}
\usepackage{float}
\usepackage{colortbl}
\usepackage{pdflscape}
\usepackage{tabu}
\usepackage{threeparttable}
\usepackage{threeparttablex}
\usepackage[normalem]{ulem}
\usepackage{makecell}
\usepackage{xcolor}

% move the hyperref stuff down here, after header-includes, to allow for - \usepackage{hyperref}

\makeatletter
\@ifpackageloaded{hyperref}{}{%
\ifxetex
  \PassOptionsToPackage{hyphens}{url}\usepackage[setpagesize=false, % page size defined by xetex
              unicode=false, % unicode breaks when used with xetex
              xetex]{hyperref}
\else
  \PassOptionsToPackage{hyphens}{url}\usepackage[draft,unicode=true]{hyperref}
\fi
}

\@ifpackageloaded{color}{
    \PassOptionsToPackage{usenames,dvipsnames}{color}
}{%
    \usepackage[usenames,dvipsnames]{color}
}
\makeatother
\hypersetup{breaklinks=true,
            bookmarks=true,
            pdfauthor={Jun Won (Lakon) Park (Group Leader) (79453940,
Group Leader, data collection, data analysis, report-writing) and Sarah
Li (60136959, data collection, data analysis, report-writing)},
             pdfkeywords = {},  
            pdftitle={White Advantage in Chess and How to Counter It},
            colorlinks=true,
            citecolor=blue,
            urlcolor=blue,
            linkcolor=magenta,
            pdfborder={0 0 0}}
\urlstyle{same}  % don't use monospace font for urls

% Add an option for endnotes. -----


% add tightlist ----------
\providecommand{\tightlist}{%
\setlength{\itemsep}{0pt}\setlength{\parskip}{0pt}}

% add some other packages ----------

% \usepackage{multicol}
% This should regulate where figures float
% See: https://tex.stackexchange.com/questions/2275/keeping-tables-figures-close-to-where-they-are-mentioned
\usepackage[section]{placeins}


\begin{document}
	
% \pagenumbering{arabic}% resets `page` counter to 1 
%    

% \maketitle

{% \usefont{T1}{pnc}{m}{n}
\setlength{\parindent}{0pt}
\thispagestyle{plain}
{\fontsize{18}{20}\selectfont\raggedright 
\maketitle  % title \par  

}

{
   \vskip 13.5pt\relax \normalsize\fontsize{11}{12} 
\textbf{\authorfont Jun Won (Lakon) Park (Group
Leader)} \hskip 15pt \emph{\small 79453940, Group Leader, data
collection, data analysis,
report-writing}   \par \textbf{\authorfont Sarah
Li} \hskip 15pt \emph{\small 60136959, data collection, data analysis,
report-writing}   

}

}








\begin{abstract}

    \hbox{\vrule height .2pt width 39.14pc}

    \vskip 8.5pt % \small 

\noindent Research question: Is white at an advantage in chess and if
so, what are some optimal strategies for black to increase their winning
probability?


    \hbox{\vrule height .2pt width 39.14pc}


\end{abstract}


\vskip -8.5pt


 % removetitleabstract

\noindent  

\hypertarget{introduction}{%
\subsubsection{Introduction}\label{introduction}}

For several centuries, millions of people worldwide have been playing
chess as a recreational and competitive board game at their homes, in
clubs, in tournaments, and even online nowadays. In the recent decades,
chess has been one of the most popular topic in machine learning and
artificial intelligence. The first move advantage has been researched
extensively since the end of 19th century, and many studies have been
shown that white has an inherent advantage.\\
\newline Although there is a general set of chess openings, less
research has been done on the effects of those openings on the final
outcome. This paper intends to confirm white's first move advantage and
study the relationship between the openings and the victory status. In
particular, we are interested in the openings that are in favour for
Black.\\
\newline This paper's data consists basic player information and game
information of over 20000 chess games played on Lichess, a very popular
internet chess platform. The data includes game length, number of turns,
winner, player elo\(^*\), all moves in Standard Chess Notation, Opening
Eco\(^*\), Opening Name, and Opening Ply\(^*\).\\
\newline Our target population is the data set itself. This paper will
perform simple random sampling and stratified random sampling from the
data set and compare the results obtained from the two different
sampling methods. We will first define a new feature called ``average
elo'' which is a mean of two player's ratings. We will define if a game
is played by beginners if the average elo of the game is below 1200.
These game records will likely negatively affect our result; if there
exists any advantage for certain side, beginners will not likely to be
able to use that advantage in their favour. \newline
----------------------------------------------------------

\begin{flushleft}
Elo : A numerical measurement to quantify a player's skill level\newline
Eco : Standardised code for any given opening\newline
Ply : Number of moves in the opening phasenewline
\end{flushleft}

\begin{center}\rule{0.5\linewidth}{0.5pt}\end{center}

\pagebreak

For every sampling method, we will perform sampling twice. In this way,
we do not have to calculate the covariance term which is difficult to
calculate as below. \[Var(X-Y) = Var(X)+Var(Y)-2Cov(X,Y)\] By combining
two independent samples, the variance will increase but we will choose a
sample size that is sufficiently large such that the margin of error is
still small. For every sampling method, we will choose the total sample
size to be 2000 which is approximately 10\% of the population. This
sample size is sufficiently large enough such that the width of our
confidence interval is small.

\hypertarget{the-parameters-of-interest}{%
\subsubsection{The parameters of
Interest}\label{the-parameters-of-interest}}

The parameters of Interest are

\begin{itemize}
\item White and Black's win rate
\item White and Black's win rate for specific openings
\item White and Black's mean number of turns to win
\end{itemize}

\hypertarget{does-white-have-higher-win-rate-than-black}{%
\subsubsection{Does White have higher win rate than
Black?}\label{does-white-have-higher-win-rate-than-black}}

We define win rate to be the proportion of games won by each side. To
confirm White's inherent advantage, we will perform two independent
sample t-test to determine whether White has a higher win rate than that
of Black. Hence, we will test the following hypothesis below
\[H_0 : p_w - p_b = 0 \quad  \quad H_a : p_w - p_b > 0\] where \(p_w\)
and \(p_b\) represent White and Black's win rate each respectively.\\
\newline We will assume equal variance among two SRS samples and conduct
t-test using pooled variance. The constructed 95\% confidence interval
is as below.

\begin{tabular}{l|r|r}
\hline
  & Win rate Estimate & SE\\
\hline
White & 0.5040 & 0.0105715\\
\hline
Black & 0.4685 & 0.0105509\\
\hline
\end{tabular}

\begin{tabular}{r|r|r|r}
\hline
Diff & Pooled SE & 95.CI.lower & 95.CI.upper\\
\hline
0.0355 & 0.0105612 & 0.0148004 & 0.0561996\\
\hline
\end{tabular}

The constructed 95\% confidence interval is \((0.0062, 0.0648)\) which
does not contain 0. Hence, we can reject the null hypothesis in favour
of the alternative hypothesis that White has a higher winning
proportion.\\
\newline We will assume equal variance among two StRS samples and
conduct t-test using pooled variance. The constructed 95\% confidence
interval is as below.

\begin{tabular}{l|l|r|r|r}
\hline
elo range & winner & win rate est. & strata size & Var. by strata\\
\hline
1200-1400 & white & 0.5217391 & 391 & 0.0006382\\
\hline
1400-1600 & white & 0.5069552 & 647 & 0.0003863\\
\hline
1600-1800 & white & 0.4898785 & 494 & 0.0005059\\
\hline
1800-2000 & white & 0.5143770 & 313 & 0.0007981\\
\hline
2000+ & white & 0.5032258 & 155 & 0.0016128\\
\hline
\end{tabular}

\begin{tabular}{l|l|r|r|r}
\hline
elo range & winner & win rate est. & strata size & Var. by strata\\
\hline
1200-1400 & black & 0.4884910 & 391 & 0.0006390\\
\hline
1400-1600 & black & 0.4544049 & 647 & 0.0003832\\
\hline
1600-1800 & black & 0.4534413 & 494 & 0.0005017\\
\hline
1800-2000 & black & 0.4423077 & 312 & 0.0007906\\
\hline
2000+ & black & 0.4358974 & 156 & 0.0015762\\
\hline
\end{tabular}

\begin{tabular}{l|r}
\hline
elo range & strata size prop.\\
\hline
1200-1400 & 0.1949283\\
\hline
1400-1600 & 0.3219334\\
\hline
1600-1800 & 0.2470221\\
\hline
1800-2000 & 0.1568638\\
\hline
2000+ & 0.0790407\\
\hline
\end{tabular}

\begin{tabular}{l|r|r}
\hline
  & Win rate Estimate & SE\\
\hline
White & 0.5063808 & 0.0078503\\
\hline
Black & 0.4573546 & 0.0081990\\
\hline
\end{tabular}

\begin{tabular}{r|r|r|r}
\hline
Diff & Pooled SE & 95.CI.lower & 95.CI.upper\\
\hline
0.0490262 & 0.0105612 & 0.0332945 & 0.0647579\\
\hline
\end{tabular}

The constructed 95\% confidence interval is \((0.0268, 0.0713)\) which
does not contain 0. Hence, we can reject the null hypothesis in favour
of the alternative hypothesis that White has a higher winning
proportion.\\
\newline

The result from both SRS and StRS show that White has a higher winning
proportion. There is no other plausible explanation for such phenomenon
other than that White has a first-move advantage. (I think the argument
is a bit weak here) \newpage

\hypertarget{what-is-an-optimal-game-opening-for-black}{%
\subsubsection{What is an optimal game opening for
Black?}\label{what-is-an-optimal-game-opening-for-black}}

Due to the many possible openings a game can start with, the sample size
in each possible domain (split by opening\_name) may be very small. In
order to ensure that the confidence interval is of reasonable width, we
will only estimate if sample size in the domain yields a confidence
interval including \(\pm 0.2\) of our estimate of win rate. In the worst
case, the win rate of Black is the same as that of White. Hence, using
\(p=0.5\), the initial minimum sample size is at least 25. Since we know
the domain size of each opening, the resulting minimum sample size for
each opening will differ and will be less than 25. \newline

\begin{tabular}{l|r|r}
\hline
opening name & n in sample 1 & n in sample 2\\
\hline
French Defense: Knight Variation & 28 & 22\\
\hline
Scandinavian Defense: Mieses-Kotroc Variation & 31 & 24\\
\hline
Scotch Game & 35 & 23\\
\hline
Sicilian Defense & 35 & 39\\
\hline
Sicilian Defense: Bowdler Attack & 41 & 32\\
\hline
Van't Kruijs Opening & 24 & 33\\
\hline
\end{tabular}

Again, using the openings above, we test the same hypothesis as above.
The resulting confidence interval is as below

\begin{Shaded}
\begin{Highlighting}[]
\NormalTok{knitr}\SpecialCharTok{::}\FunctionTok{kable}\NormalTok{(openings)}
\end{Highlighting}
\end{Shaded}

\begin{tabular}{l|r|r|r|r}
\hline
opening name & Diff. win rate & SE & 95.CI.lower & 95.CI.upper\\
\hline
French Defense: Knight Variation & -0.0909091 & 0.0942050 & -0.2803210 & 0.0985028\\
\hline
Scandinavian Defense: Mieses-Kotroc Variation & 0.3346774 & 0.0842921 & 0.1656088 & 0.5037460\\
\hline
Scotch Game & 0.1962733 & 0.0820415 & 0.0319243 & 0.3606223\\
\hline
Sicilian Defense & -0.1355311 & 0.0768343 & -0.2886975 & 0.0176352\\
\hline
Sicilian Defense: Bowdler Attack & -0.0228659 & 0.0764140 & -0.1752310 & 0.1294993\\
\hline
Van't Kruijs Opening & -0.1704545 & 0.0874960 & -0.3458004 & 0.0048913\\
\hline
\end{tabular}

Using SRS sampling and assuming equal variances, all openings but
``Scandinavian Defense: Mieses-Kotroc Variation'' and ``Scotch Game''
contain 0 in their 95\% confidence intervals. For these openings, we
cannot reject the null hypothesis which states that there is no
difference in win rates. More specifically, there is insufficient
evidence to suggest that black has a higher win rate when these openings
are used. For the intervals that are strictly greater than 0, there is
sufficient evidence to reject the null hypothesis in favour of the
alternative hypothesis that White has higher win rate. In all, there is
no opening that yields higher win rate for Black.\\
\newline Using StRS, we will again determine the valid openings in two
different samples and test the same hypothesis.

\begin{tabular}{l|r|r}
\hline
opening name & n in sample 1 & n in sample 2\\
\hline
French Defense: Knight Variation & 28 & 22\\
\hline
Scandinavian Defense: Mieses-Kotroc Variation & 31 & 24\\
\hline
Scotch Game & 35 & 23\\
\hline
Sicilian Defense & 35 & 39\\
\hline
Sicilian Defense: Bowdler Attack & 41 & 32\\
\hline
Van't Kruijs Opening & 24 & 33\\
\hline
\end{tabular}

The resulting confidence interval for the opening above is as follows

\begin{tabular}{l|r|r|r|r}
\hline
opening name & Diff. win rate & SE & 95.CI.lower & 95.CI.upper\\
\hline
French Defense: Knight Variation & 0.0426424 & 0.0651798 & -0.0879285 & 0.1732133\\
\hline
Scandinavian Defense: Mieses-Kotroc Variation & 0.2094644 & 0.0495076 & 0.1103999 & 0.3085289\\
\hline
Scotch Game & -0.0172765 & 0.0705189 & -0.1586583 & 0.1241054\\
\hline
Sicilian Defense & -0.1609474 & 0.0549505 & -0.2702818 & -0.0516131\\
\hline
Sicilian Defense: Bowdler Attack & -0.1600468 & 0.0686168 & -0.2978677 & -0.0222258\\
\hline
Van't Kruijs Opening & -0.3068106 & 0.0539687 & -0.4152100 & -0.1984112\\
\hline
\end{tabular}

The confidence intervals for openings ``Sicilian Defense'', ``Sicilian
Defense: Bowdler Attack'', and '' Van't Kruijs Opening'' are strictly
negative. For these openings, Black has a higher win rate than that of
White. For ``Scandinavian Defense: Mieses-Kotroc Variation'', the
constructed confidence interval is strictly positive so White has a
higher win rate than that of Black. For the openings with confidence
intervals that contain 0, we cannot reject the null hypothesis which
states that there is no difference in win rates. More specifically,
there is insufficient evidence to suggest that black has a higher win
rate when these openings are used.\\
\newline Since SRS and StRS sample contain different valid openings, it
is not possible to compare the results. However, we have found several
openings in which Black has a higher win rate.

\hypertarget{does-it-take-longer-for-black-to-win}{%
\subsubsection{Does it take longer for Black to
win?}\label{does-it-take-longer-for-black-to-win}}

We have already confirmed that White has a first-move advantage over
Black. So how does Black actually overcome this advantage? Our
hypothesis is that Black will need to spend extra turns to overcome the
disadvantage in the beginning. This leads to increase an increase in
overall turn spent by Black to win. Hence, we will test the following
hypothesis,
\[H_0 : \mu_W - \mu_B = 0 \quad \quad H_a: \mu_W - \mu_B < 0\] where
\(\mu_W\) and \(\mu_B\) represent White and Black's mean number of turns
to win each respectively.\\
\newline We will assume equal variance among two SRS samples and conduct
t-test using pooled variance. The constructed 95\% confidence interval
is as below.

\begin{tabular}{r|r}
\hline
Est. & SE\\
\hline
57.68849 & 0.9787740\\
\hline
60.52721 & 0.9840654\\
\hline
\end{tabular}

\begin{tabular}{r|r|r|r}
\hline
-2.838722 & 0.9814232 & -4.762277 & -0.9151682\\
\hline
\end{tabular}

The constructed confidence interval does not contain 0 which implies
there is sufficient evidence to reject the null hypothesis in favour of
the alternative hypothesis that Black takes longer to win.\\
\newline Using StRS and assuming equal variance,

\begin{tabular}{l|r}
\hline
elo range & mean turns for white\\
\hline
1200-1400 & 51.26471\\
\hline
1400-1600 & 55.52744\\
\hline
1600-1800 & 60.51653\\
\hline
1800-2000 & 65.25466\\
\hline
2000+ & 55.08974\\
\hline
\end{tabular}

\begin{tabular}{l|r}
\hline
elo range & mean turns for black\\
\hline
1200-1400 & 55.90052\\
\hline
1400-1600 & 57.77551\\
\hline
1600-1800 & 60.82143\\
\hline
1800-2000 & 65.34783\\
\hline
2000+ & 72.05882\\
\hline
\end{tabular}

\begin{tabular}{l|r|r|l}
\hline
elo range & strata size & var. by strata & winner\\
\hline
1200-1400 & 391 & 1.725542 & white\\
\hline
1400-1600 & 647 & 1.235008 & white\\
\hline
1600-1800 & 494 & 1.563627 & white\\
\hline
1800-2000 & 313 & 3.240864 & white\\
\hline
2000+ & 155 & 4.484618 & white\\
\hline
\end{tabular}

\begin{tabular}{l|r|r|l}
\hline
elo range & strata size & var. by strata & winner\\
\hline
1200-1400 & 391 & 2.331782 & black\\
\hline
1400-1600 & 647 & 1.338437 & black\\
\hline
1600-1800 & 494 & 1.553842 & black\\
\hline
1800-2000 & 312 & 3.169745 & black\\
\hline
2000+ & 156 & 6.058140 & black\\
\hline
\end{tabular}

\begin{tabular}{r|r|r|r}
\hline
Diff & Pooled SE & 95.CI.lower & 95.CI.upper\\
\hline
-3.058561 & 0.0490262 & -4.687201 & -1.42992\\
\hline
\end{tabular}

The constructed confidence interval does not contain 0 which implies
there is sufficient evidence to reject the null hypothesis in favour of
the alternative hypothesis that Black takes longer to win.\\
\newline

\hypertarget{conclusion}{%
\subsubsection{Conclusion}\label{conclusion}}

something

\newpage
\begin{center}
\Large{Appendix}
\end{center}

\begin{Shaded}
\begin{Highlighting}[]
\CommentTok{\# Load data}
\NormalTok{df }\OtherTok{\textless{}{-}} \FunctionTok{read.csv}\NormalTok{(}\StringTok{"games.csv"}\NormalTok{)}

\CommentTok{\# Calculate the average elo of the game}
\NormalTok{df }\OtherTok{\textless{}{-}} \FunctionTok{mutate}\NormalTok{(df }\SpecialCharTok{\%\textgreater{}\%} \FunctionTok{rowwise}\NormalTok{(),}
       \AttributeTok{average\_elo =} \FunctionTok{rowMeans}\NormalTok{(}\FunctionTok{cbind}\NormalTok{(black\_rating, white\_rating)))}

\CommentTok{\# Filter games by average elo}
\NormalTok{df }\OtherTok{\textless{}{-}} \FunctionTok{filter}\NormalTok{(df, average\_elo }\SpecialCharTok{\textgreater{}=} \DecValTok{1200}\NormalTok{)}

\CommentTok{\# Select only necessary columns for analysis}
\NormalTok{df }\OtherTok{\textless{}{-}} \FunctionTok{subset}\NormalTok{(df, }
             \AttributeTok{select =} \FunctionTok{c}\NormalTok{(id, turns, white\_rating, black\_rating, victory\_status, }
\NormalTok{                        winner, moves, opening\_eco, opening\_name, opening\_ply, average\_elo ))}
\end{Highlighting}
\end{Shaded}

\begin{Shaded}
\begin{Highlighting}[]
\CommentTok{\# Simple Random Sampling}
\NormalTok{N }\OtherTok{\textless{}{-}} \FunctionTok{nrow}\NormalTok{(df)}
\NormalTok{n }\OtherTok{\textless{}{-}} \DecValTok{2000}
\FunctionTok{set.seed}\NormalTok{(}\DecValTok{1234}\NormalTok{)}
\NormalTok{sample.index.s1 }\OtherTok{\textless{}{-}} \FunctionTok{sample}\NormalTok{(}\DecValTok{1}\SpecialCharTok{:}\NormalTok{N, }\AttributeTok{size=}\NormalTok{n, }\AttributeTok{replace =} \ConstantTok{FALSE}\NormalTok{)}
\NormalTok{srs.sample.s1 }\OtherTok{\textless{}{-}}\NormalTok{ df[sample.index.s1,]}

\FunctionTok{set.seed}\NormalTok{(}\DecValTok{4321}\NormalTok{)}
\NormalTok{sample.index.s2 }\OtherTok{\textless{}{-}} \FunctionTok{sample}\NormalTok{(}\DecValTok{1}\SpecialCharTok{:}\NormalTok{N, }\AttributeTok{size=}\NormalTok{n, }\AttributeTok{replace =} \ConstantTok{FALSE}\NormalTok{)}
\NormalTok{srs.sample.s2 }\OtherTok{\textless{}{-}}\NormalTok{ df[sample.index.s2,]}
\end{Highlighting}
\end{Shaded}

\begin{Shaded}
\begin{Highlighting}[]
\CommentTok{\# Determine minimum and maximum before stratifying}
\FunctionTok{min}\NormalTok{(df}\SpecialCharTok{$}\NormalTok{average\_elo)}
\FunctionTok{max}\NormalTok{(df}\SpecialCharTok{$}\NormalTok{average\_elo)}

\NormalTok{df}\SpecialCharTok{$}\NormalTok{elo\_range }\OtherTok{\textless{}{-}} \FunctionTok{cut}\NormalTok{(df}\SpecialCharTok{$}\NormalTok{average\_elo,}
                    \FunctionTok{c}\NormalTok{(}\DecValTok{1200}\NormalTok{, }\DecValTok{1400}\NormalTok{, }\DecValTok{1600}\NormalTok{, }\DecValTok{1800}\NormalTok{, }\DecValTok{2000}\NormalTok{, }\DecValTok{2600}\NormalTok{))}
\FunctionTok{levels}\NormalTok{(df}\SpecialCharTok{$}\NormalTok{elo\_range) }\OtherTok{\textless{}{-}} \FunctionTok{c}\NormalTok{(}\StringTok{"1200{-}1400"}\NormalTok{, }\StringTok{"1400{-}1600"}\NormalTok{, }\StringTok{"1600{-}1800"}\NormalTok{, }\StringTok{"1800{-}2000"}\NormalTok{,}
                          \StringTok{"2000+"}\NormalTok{)}
\NormalTok{df}\SpecialCharTok{$}\NormalTok{winner }\OtherTok{\textless{}{-}} \FunctionTok{as.factor}\NormalTok{(df}\SpecialCharTok{$}\NormalTok{winner)}

\CommentTok{\# Check if standard deviations of the strata are identical}
\NormalTok{se.by.strata }\OtherTok{\textless{}{-}} \FunctionTok{aggregate}\NormalTok{(}\FunctionTok{as.numeric}\NormalTok{(df}\SpecialCharTok{$}\NormalTok{winner), }\AttributeTok{by=}\FunctionTok{list}\NormalTok{(df}\SpecialCharTok{$}\NormalTok{elo\_range), }\AttributeTok{FUN=}\NormalTok{sd)}
\NormalTok{se.by.strata}

\CommentTok{\# Standard deviations within strata are not identical, \textbackslash{}}
\CommentTok{\# so find optimal sample sizes}
\NormalTok{pop.size.by.strata }\OtherTok{\textless{}{-}} \FunctionTok{aggregate}\NormalTok{(df}\SpecialCharTok{$}\NormalTok{winner, }\AttributeTok{by=}\FunctionTok{list}\NormalTok{(df}\SpecialCharTok{$}\NormalTok{elo\_range), }\AttributeTok{FUN=}\NormalTok{length)}
\NormalTok{denom }\OtherTok{\textless{}{-}} \FunctionTok{sum}\NormalTok{(pop.size.by.strata[}\DecValTok{2}\NormalTok{] }\SpecialCharTok{*}\NormalTok{ se.by.strata[}\DecValTok{2}\NormalTok{])}
\NormalTok{sample.size.by.strata }\OtherTok{\textless{}{-}}\NormalTok{ (pop.size.by.strata[}\DecValTok{2}\NormalTok{] }\SpecialCharTok{*}\NormalTok{ se.by.strata[}\DecValTok{2}\NormalTok{]) }\SpecialCharTok{/}\NormalTok{ denom}

\CommentTok{\# Sample from each strata}
\NormalTok{strsample }\OtherTok{\textless{}{-}} \ControlFlowTok{function}\NormalTok{(df, sample.size.by.strata, n, seed) \{}
\NormalTok{  str.sample }\OtherTok{\textless{}{-}}\NormalTok{ df[}\ConstantTok{FALSE}\NormalTok{,]}
  \FunctionTok{colnames}\NormalTok{(str.sample) }\OtherTok{\textless{}{-}} \FunctionTok{names}\NormalTok{(df)}
  \ControlFlowTok{for}\NormalTok{ (i }\ControlFlowTok{in} \DecValTok{1}\SpecialCharTok{:}\FunctionTok{length}\NormalTok{(}\FunctionTok{levels}\NormalTok{(df}\SpecialCharTok{$}\NormalTok{elo\_range))) \{}
\NormalTok{    strata }\OtherTok{\textless{}{-}} \FunctionTok{which}\NormalTok{(df}\SpecialCharTok{$}\NormalTok{elo\_range }\SpecialCharTok{==} \FunctionTok{levels}\NormalTok{(df}\SpecialCharTok{$}\NormalTok{elo\_range)[i])}
    \FunctionTok{set.seed}\NormalTok{(seed)}
\NormalTok{    sample.idx }\OtherTok{\textless{}{-}} \FunctionTok{sample}\NormalTok{(strata, }
                             \AttributeTok{size =} \FunctionTok{ceiling}\NormalTok{(sample.size.by.strata}\SpecialCharTok{$}\NormalTok{x[i] }\SpecialCharTok{*}\NormalTok{ n), }
                             \AttributeTok{replace =} \ConstantTok{FALSE}\NormalTok{)}
\NormalTok{    sample }\OtherTok{\textless{}{-}}\NormalTok{ df[sample.idx,]}
\NormalTok{    str.sample }\OtherTok{\textless{}{-}} \FunctionTok{rbind}\NormalTok{(str.sample, sample)}
\NormalTok{  \}}
  
  \CommentTok{\# Stratified sample contains 1003 samples due to rounding of the proportions,}
  \CommentTok{\# so we randomly remove three from random strata}
\NormalTok{  strata.for.removal }\OtherTok{\textless{}{-}} \FunctionTok{sample}\NormalTok{(}\DecValTok{1}\SpecialCharTok{:}\DecValTok{5}\NormalTok{, }\DecValTok{2}\NormalTok{)}
  \ControlFlowTok{for}\NormalTok{ (s }\ControlFlowTok{in}\NormalTok{ strata.for.removal) \{}
    \FunctionTok{set.seed}\NormalTok{(}\DecValTok{1234}\NormalTok{)}
\NormalTok{    to.remove }\OtherTok{\textless{}{-}} \FunctionTok{sample}\NormalTok{(}\FunctionTok{which}\NormalTok{(str.sample}\SpecialCharTok{$}\NormalTok{elo\_range }\SpecialCharTok{==} \FunctionTok{levels}\NormalTok{(df}\SpecialCharTok{$}\NormalTok{elo\_range)[s]), }\DecValTok{1}\NormalTok{)}
\NormalTok{    str.sample }\OtherTok{\textless{}{-}}\NormalTok{ str.sample[}\SpecialCharTok{{-}}\NormalTok{to.remove,]}
\NormalTok{  \}}
  
  \FunctionTok{return}\NormalTok{(str.sample)}
\NormalTok{\}}

\NormalTok{white.str.sample }\OtherTok{\textless{}{-}} \FunctionTok{strsample}\NormalTok{(df, sample.size.by.strata, n, }\DecValTok{1234}\NormalTok{) }\SpecialCharTok{\%\textgreater{}\%} \FunctionTok{group\_by}\NormalTok{(elo\_range) }
\NormalTok{black.str.sample }\OtherTok{\textless{}{-}} \FunctionTok{strsample}\NormalTok{(df, sample.size.by.strata, n, }\DecValTok{4321}\NormalTok{) }\SpecialCharTok{\%\textgreater{}\%} \FunctionTok{group\_by}\NormalTok{(elo\_range) }
\end{Highlighting}
\end{Shaded}

\begin{Shaded}
\begin{Highlighting}[]
\NormalTok{z}\FloatTok{.95} \OtherTok{\textless{}{-}} \FunctionTok{qnorm}\NormalTok{(}\FloatTok{0.975}\NormalTok{)}
\CommentTok{\# Returns the sample variance of a given proportion}
\NormalTok{var.est }\OtherTok{\textless{}{-}} \ControlFlowTok{function}\NormalTok{(p) \{}
\NormalTok{  p }\SpecialCharTok{*}\NormalTok{ (}\DecValTok{1} \SpecialCharTok{{-}}\NormalTok{ p)}
\NormalTok{\}}
\CommentTok{\# Calculate white\textquotesingle{}s win rate}
\NormalTok{white.prop }\OtherTok{\textless{}{-}}\NormalTok{ srs.sample.s1 }\SpecialCharTok{\%\textgreater{}\%}
  \FunctionTok{count}\NormalTok{(winner) }\SpecialCharTok{\%\textgreater{}\%}
  \FunctionTok{group\_by}\NormalTok{(winner) }\SpecialCharTok{\%\textgreater{}\%}
  \FunctionTok{mutate}\NormalTok{(}\AttributeTok{win.prop =}\NormalTok{ n }\SpecialCharTok{/} \DecValTok{2000}\NormalTok{)}

\NormalTok{white.p }\OtherTok{\textless{}{-}} \FunctionTok{as.numeric}\NormalTok{(white.prop[}\DecValTok{3}\NormalTok{,}\DecValTok{3}\NormalTok{])}

\NormalTok{black.prop }\OtherTok{\textless{}{-}}\NormalTok{ srs.sample.s2 }\SpecialCharTok{\%\textgreater{}\%}
  \FunctionTok{count}\NormalTok{(winner) }\SpecialCharTok{\%\textgreater{}\%}
  \FunctionTok{group\_by}\NormalTok{(winner) }\SpecialCharTok{\%\textgreater{}\%}
  \FunctionTok{mutate}\NormalTok{(}\AttributeTok{win.prop =}\NormalTok{ n }\SpecialCharTok{/} \DecValTok{2000}\NormalTok{)}

\NormalTok{black.p }\OtherTok{\textless{}{-}} \FunctionTok{as.numeric}\NormalTok{(black.prop[}\DecValTok{1}\NormalTok{,}\DecValTok{3}\NormalTok{])}

\NormalTok{srs.se }\OtherTok{\textless{}{-}} \FunctionTok{sqrt}\NormalTok{((}\DecValTok{1}\SpecialCharTok{{-}}\NormalTok{n}\SpecialCharTok{/}\NormalTok{N)}\SpecialCharTok{*}\NormalTok{(}\FunctionTok{var.est}\NormalTok{(white.p) }\SpecialCharTok{+} \FunctionTok{var.est}\NormalTok{(black.p))}\SpecialCharTok{/}\NormalTok{n)}


\NormalTok{table1 }\OtherTok{\textless{}{-}} \FunctionTok{t}\NormalTok{(}\FunctionTok{c}\NormalTok{(white.p, black.p, srs.se, (white.p}\SpecialCharTok{{-}}\NormalTok{black.p) }\SpecialCharTok{+}\NormalTok{ z}\FloatTok{.95} \SpecialCharTok{*}\NormalTok{ srs.se }\SpecialCharTok{*} \FunctionTok{c}\NormalTok{(}\SpecialCharTok{{-}}\DecValTok{1}\NormalTok{, }\DecValTok{1}\NormalTok{)))}
\NormalTok{table1 }\OtherTok{\textless{}{-}} \FunctionTok{data.frame}\NormalTok{(table1)}
\FunctionTok{names}\NormalTok{(table1) }\OtherTok{\textless{}{-}} \FunctionTok{c}\NormalTok{(}\StringTok{"White\textquotesingle{}s win rate"}\NormalTok{, }\StringTok{"Black\textquotesingle{}s win rate"}\NormalTok{, }\StringTok{"SE of the difference"}\NormalTok{, }\StringTok{"Lower Confidence Interval"}\NormalTok{, }\StringTok{"Upper Confidence Interval"}\NormalTok{)}
\CommentTok{\#knitr::kable(table1)}
\end{Highlighting}
\end{Shaded}

\begin{Shaded}
\begin{Highlighting}[]
\CommentTok{\# Calculate Nh/N, the strata proportion}
\NormalTok{Nh }\OtherTok{\textless{}{-}}\NormalTok{ df }\SpecialCharTok{\%\textgreater{}\%} \FunctionTok{count}\NormalTok{(elo\_range, }\AttributeTok{.drop=}\ConstantTok{FALSE}\NormalTok{)}
\NormalTok{Nh }\OtherTok{\textless{}{-}}\NormalTok{ Nh[}\FunctionTok{complete.cases}\NormalTok{(Nh),]}

\NormalTok{nh.white }\OtherTok{\textless{}{-}}\NormalTok{ white.str.sample }\SpecialCharTok{\%\textgreater{}\%} \FunctionTok{count}\NormalTok{(elo\_range, }\AttributeTok{.drop=}\ConstantTok{FALSE}\NormalTok{)}
\NormalTok{nh.black }\OtherTok{\textless{}{-}}\NormalTok{ black.str.sample }\SpecialCharTok{\%\textgreater{}\%} \FunctionTok{count}\NormalTok{(elo\_range, }\AttributeTok{.drop=}\ConstantTok{FALSE}\NormalTok{)}
\NormalTok{strata.size.prop }\OtherTok{\textless{}{-}}\NormalTok{ Nh[}\DecValTok{2}\NormalTok{] }\SpecialCharTok{/}\NormalTok{ N}

\CommentTok{\# Calculate white\textquotesingle{}s win proportion by each strata}
\NormalTok{white.win.prop }\OtherTok{\textless{}{-}}\NormalTok{ white.str.sample }\SpecialCharTok{\%\textgreater{}\%}
  \FunctionTok{count}\NormalTok{(winner) }\SpecialCharTok{\%\textgreater{}\%}
  \FunctionTok{group\_by}\NormalTok{(elo\_range) }\SpecialCharTok{\%\textgreater{}\%}
  \FunctionTok{mutate}\NormalTok{(}\AttributeTok{win.prop =}\NormalTok{ n }\SpecialCharTok{/} \FunctionTok{sum}\NormalTok{(n))}

\CommentTok{\# The estimated aggregated win proportion for white}
\NormalTok{white.prop }\OtherTok{\textless{}{-}}\NormalTok{ white.win.prop[white.win.prop}\SpecialCharTok{$}\NormalTok{winner }\SpecialCharTok{==} \StringTok{"white"}\NormalTok{, ] }
\NormalTok{white.p.str.est }\OtherTok{\textless{}{-}} \FunctionTok{sum}\NormalTok{(white.prop}\SpecialCharTok{$}\NormalTok{win.prop }\SpecialCharTok{*}\NormalTok{ strata.size.prop)}

\CommentTok{\# The estimated aggregated variance of win proportion for white}
\NormalTok{white.se.by.strata }\OtherTok{\textless{}{-}} \FunctionTok{bind\_cols}\NormalTok{(white.prop, }\AttributeTok{nh =}\NormalTok{ nh.white}\SpecialCharTok{$}\NormalTok{n)}
\NormalTok{white.se.by.strata }\OtherTok{\textless{}{-}}\NormalTok{ white.se.by.strata }\SpecialCharTok{\%\textgreater{}\%} \FunctionTok{mutate}\NormalTok{(}\AttributeTok{var.by.strata =}\NormalTok{ win.prop }\SpecialCharTok{*}\NormalTok{ (}\DecValTok{1}\SpecialCharTok{{-}}\NormalTok{win.prop)}\SpecialCharTok{/}\NormalTok{nh)}
\NormalTok{white.se.by.strata }\OtherTok{\textless{}{-}} \FunctionTok{bind\_cols}\NormalTok{(white.se.by.strata, }\AttributeTok{strata.prop.sq =}\NormalTok{ strata.size.prop}\SpecialCharTok{$}\NormalTok{n}\SpecialCharTok{\^{}}\DecValTok{2}\NormalTok{)}
\NormalTok{white.se.by.strata }\OtherTok{\textless{}{-}}\NormalTok{ white.se.by.strata }\SpecialCharTok{\%\textgreater{}\%} \FunctionTok{mutate}\NormalTok{(strata.prop.sq}\SpecialCharTok{*}\NormalTok{(}\DecValTok{1}\SpecialCharTok{{-}}\NormalTok{n}\SpecialCharTok{/}\NormalTok{nh)}\SpecialCharTok{*}\NormalTok{var.by.strata)}
\NormalTok{white.str.se }\OtherTok{\textless{}{-}} \FunctionTok{sqrt}\NormalTok{(}\FunctionTok{sum}\NormalTok{(white.se.by.strata}\SpecialCharTok{$}\StringTok{\textasciigrave{}}\AttributeTok{strata.prop.sq * (1 {-} n/nh) * var.by.strata}\StringTok{\textasciigrave{}}\NormalTok{))}

\CommentTok{\# Calculate black\textquotesingle{}s win proportion by each strata}
\NormalTok{black.win.prop }\OtherTok{\textless{}{-}}\NormalTok{ black.str.sample }\SpecialCharTok{\%\textgreater{}\%}
  \FunctionTok{count}\NormalTok{(winner) }\SpecialCharTok{\%\textgreater{}\%}
  \FunctionTok{group\_by}\NormalTok{(elo\_range) }\SpecialCharTok{\%\textgreater{}\%}
  \FunctionTok{mutate}\NormalTok{(}\AttributeTok{win.prop =}\NormalTok{ n }\SpecialCharTok{/} \FunctionTok{sum}\NormalTok{(n))}

\CommentTok{\# The estimated aggregated win proportion for white}
\NormalTok{black.prop }\OtherTok{\textless{}{-}}\NormalTok{ black.win.prop[black.win.prop}\SpecialCharTok{$}\NormalTok{winner }\SpecialCharTok{==} \StringTok{"black"}\NormalTok{, ] }
\NormalTok{black.p.str.est }\OtherTok{\textless{}{-}} \FunctionTok{sum}\NormalTok{(black.prop}\SpecialCharTok{$}\NormalTok{win.prop }\SpecialCharTok{*}\NormalTok{ strata.size.prop)}

\CommentTok{\# The estimated aggregated variance of win proportion for black}
\NormalTok{black.se.by.strata }\OtherTok{\textless{}{-}} \FunctionTok{bind\_cols}\NormalTok{(black.prop, }\AttributeTok{nh =}\NormalTok{ nh.black}\SpecialCharTok{$}\NormalTok{n)}
\NormalTok{black.se.by.strata }\OtherTok{\textless{}{-}}\NormalTok{ black.se.by.strata }\SpecialCharTok{\%\textgreater{}\%} \FunctionTok{mutate}\NormalTok{(}\AttributeTok{var.by.strata =}\NormalTok{ win.prop }\SpecialCharTok{*}\NormalTok{ (}\DecValTok{1}\SpecialCharTok{{-}}\NormalTok{win.prop)}\SpecialCharTok{/}\NormalTok{nh)}
\NormalTok{black.se.by.strata }\OtherTok{\textless{}{-}} \FunctionTok{bind\_cols}\NormalTok{(black.se.by.strata, }\AttributeTok{strata.prop.sq =}\NormalTok{ strata.size.prop}\SpecialCharTok{$}\NormalTok{n}\SpecialCharTok{\^{}}\DecValTok{2}\NormalTok{)}
\NormalTok{black.se.by.strata }\OtherTok{\textless{}{-}}\NormalTok{ black.se.by.strata }\SpecialCharTok{\%\textgreater{}\%} \FunctionTok{mutate}\NormalTok{(strata.prop.sq}\SpecialCharTok{*}\NormalTok{(}\DecValTok{1}\SpecialCharTok{{-}}\NormalTok{n}\SpecialCharTok{/}\NormalTok{nh)}\SpecialCharTok{*}\NormalTok{var.by.strata)}
\NormalTok{black.str.se }\OtherTok{\textless{}{-}} \FunctionTok{sqrt}\NormalTok{(}\FunctionTok{sum}\NormalTok{(black.se.by.strata}\SpecialCharTok{$}\StringTok{\textasciigrave{}}\AttributeTok{strata.prop.sq * (1 {-} n/nh) * var.by.strata}\StringTok{\textasciigrave{}}\NormalTok{))}

\CommentTok{\# Their difference}
\NormalTok{diff.p }\OtherTok{\textless{}{-}}\NormalTok{ white.p.str.est }\SpecialCharTok{{-}}\NormalTok{ black.p.str.est}
\NormalTok{diff.se }\OtherTok{\textless{}{-}} \FunctionTok{sqrt}\NormalTok{(white.str.se}\SpecialCharTok{\^{}}\DecValTok{2} \SpecialCharTok{+}\NormalTok{ black.str.se}\SpecialCharTok{\^{}}\DecValTok{2}\NormalTok{)}
\NormalTok{(diff.p) }\SpecialCharTok{+}\NormalTok{ z}\FloatTok{.95} \SpecialCharTok{*}\NormalTok{ diff.se }\SpecialCharTok{*} \FunctionTok{c}\NormalTok{(}\SpecialCharTok{{-}}\DecValTok{1}\NormalTok{, }\DecValTok{1}\NormalTok{)}
\end{Highlighting}
\end{Shaded}

\begin{Shaded}
\begin{Highlighting}[]
\CommentTok{\# Guess the most conservative variance}
\CommentTok{\# Find minimum domain sample size for desired CI width}
\NormalTok{var.guess }\OtherTok{\textless{}{-}} \FloatTok{0.25}
\NormalTok{ci.width }\OtherTok{\textless{}{-}} \FloatTok{0.2}
\NormalTok{n0 }\OtherTok{\textless{}{-}}\NormalTok{ z}\FloatTok{.95}\SpecialCharTok{\^{}}\DecValTok{2} \SpecialCharTok{*}\NormalTok{ var.guess }\SpecialCharTok{/}\NormalTok{ ci.width}\SpecialCharTok{\^{}}\DecValTok{2}
\end{Highlighting}
\end{Shaded}

\begin{Shaded}
\begin{Highlighting}[]
\NormalTok{openings.df.s1 }\OtherTok{\textless{}{-}} \FunctionTok{data.frame}\NormalTok{(}\FunctionTok{table}\NormalTok{(white.str.sample}\SpecialCharTok{$}\NormalTok{opening\_name))}
\NormalTok{openings.df.s2 }\OtherTok{\textless{}{-}} \FunctionTok{data.frame}\NormalTok{(}\FunctionTok{table}\NormalTok{(black.str.sample}\SpecialCharTok{$}\NormalTok{opening\_name))}
\FunctionTok{names}\NormalTok{(openings.df.s1) }\OtherTok{\textless{}{-}} \FunctionTok{c}\NormalTok{(}\StringTok{"name"}\NormalTok{, }\StringTok{"frequency"}\NormalTok{)}
\FunctionTok{names}\NormalTok{(openings.df.s2) }\OtherTok{\textless{}{-}} \FunctionTok{c}\NormalTok{(}\StringTok{"name"}\NormalTok{, }\StringTok{"frequency"}\NormalTok{)}

\CommentTok{\# Include openings with sample size large enough for usable CI}
\NormalTok{openings.freq.s1 }\OtherTok{\textless{}{-}}\NormalTok{ openings.df.s1[openings.df.s1}\SpecialCharTok{$}\NormalTok{frequency }\SpecialCharTok{\textgreater{}} \DecValTok{15}\NormalTok{,]}
\NormalTok{openings.freq.s2 }\OtherTok{\textless{}{-}}\NormalTok{ openings.df.s2[openings.df.s2}\SpecialCharTok{$}\NormalTok{frequency }\SpecialCharTok{\textgreater{}} \DecValTok{15}\NormalTok{,]}

\NormalTok{openings.df.p }\OtherTok{\textless{}{-}} \FunctionTok{data.frame}\NormalTok{(}\FunctionTok{table}\NormalTok{(df}\SpecialCharTok{$}\NormalTok{opening\_name))}
\FunctionTok{names}\NormalTok{(openings.df.p) }\OtherTok{\textless{}{-}} \FunctionTok{c}\NormalTok{(}\StringTok{"name"}\NormalTok{, }\StringTok{"frequency"}\NormalTok{)}

\CommentTok{\# openings.size.p1 \textless{}{-} openings.df.p[openings.df.p$name \%in\% openings.freq.s1$name,]}
\CommentTok{\# openings.size.p2 \textless{}{-} openings.df.p[openings.df.p$name \%in\% openings.freq.s2$name,]}

\CommentTok{\# Include openings with sample sizes yielding the desired CI width}
\NormalTok{domain.sizes.s1 }\OtherTok{\textless{}{-}} \FunctionTok{c}\NormalTok{()}
\NormalTok{domain.sizes.s2 }\OtherTok{\textless{}{-}} \FunctionTok{c}\NormalTok{()}

\ControlFlowTok{for}\NormalTok{ (name }\ControlFlowTok{in}\NormalTok{ openings.freq.s1}\SpecialCharTok{$}\NormalTok{name) \{}
\NormalTok{  domain.sizes.s1 }\OtherTok{\textless{}{-}} \FunctionTok{c}\NormalTok{(domain.sizes.s1, n0 }\SpecialCharTok{/}\NormalTok{ (}\DecValTok{1} \SpecialCharTok{+}\NormalTok{ n0 }\SpecialCharTok{/}\NormalTok{ openings.df.p[openings.df.p}\SpecialCharTok{$}\NormalTok{name }\SpecialCharTok{==}\NormalTok{ name,]}\SpecialCharTok{$}\NormalTok{frequency))}
\NormalTok{\}}

\ControlFlowTok{for}\NormalTok{ (name }\ControlFlowTok{in}\NormalTok{ openings.freq.s2}\SpecialCharTok{$}\NormalTok{name) \{}
\NormalTok{  domain.sizes.s2 }\OtherTok{\textless{}{-}} \FunctionTok{c}\NormalTok{(domain.sizes.s2, n0 }\SpecialCharTok{/}\NormalTok{ (}\DecValTok{1} \SpecialCharTok{+}\NormalTok{ n0 }\SpecialCharTok{/}\NormalTok{ openings.df.p[openings.df.p}\SpecialCharTok{$}\NormalTok{name }\SpecialCharTok{==}\NormalTok{ name,]}\SpecialCharTok{$}\NormalTok{frequency))}
\NormalTok{\}}

\NormalTok{openings.valid.s1 }\OtherTok{\textless{}{-}}\NormalTok{ openings.freq.s1[openings.freq.s1}\SpecialCharTok{$}\NormalTok{frequency }\SpecialCharTok{\textgreater{}}\NormalTok{ domain.sizes.s1,]}
\NormalTok{openings.valid.s2 }\OtherTok{\textless{}{-}}\NormalTok{ openings.freq.s2[openings.freq.s2}\SpecialCharTok{$}\NormalTok{frequency }\SpecialCharTok{\textgreater{}}\NormalTok{ domain.sizes.s2,]}

\NormalTok{openings.valid.str.sample }\OtherTok{\textless{}{-}} \FunctionTok{merge}\NormalTok{(openings.valid.s1, openings.valid.s2, }\AttributeTok{by =} \StringTok{"name"}\NormalTok{)}
\end{Highlighting}
\end{Shaded}

\begin{Shaded}
\begin{Highlighting}[]
\NormalTok{estimates }\OtherTok{\textless{}{-}} \FunctionTok{rep}\NormalTok{(}\DecValTok{0}\NormalTok{, }\FunctionTok{nrow}\NormalTok{(openings.valid.str.sample))}
\NormalTok{intervals }\OtherTok{\textless{}{-}} \FunctionTok{matrix}\NormalTok{(}\DecValTok{0}\NormalTok{, }\FunctionTok{nrow}\NormalTok{(openings.valid.str.sample), }\DecValTok{2}\NormalTok{)}
\ControlFlowTok{for}\NormalTok{ (i }\ControlFlowTok{in} \DecValTok{1}\SpecialCharTok{:}\FunctionTok{nrow}\NormalTok{(openings.valid.str.sample)) \{}
  \CommentTok{\# Find estimate and CI for difference in win rate for white/black}
  \CommentTok{\# for one domain}
\NormalTok{  domain.name }\OtherTok{\textless{}{-}}\NormalTok{ openings.valid.str.sample[i, }\DecValTok{1}\NormalTok{]}
\NormalTok{  domain.s1 }\OtherTok{\textless{}{-}}\NormalTok{ white.str.sample[white.str.sample}\SpecialCharTok{$}\NormalTok{opening\_name }\SpecialCharTok{==}\NormalTok{ domain.name,]}
\NormalTok{  domain.s2 }\OtherTok{\textless{}{-}}\NormalTok{ black.str.sample[black.str.sample}\SpecialCharTok{$}\NormalTok{opening\_name }\SpecialCharTok{==}\NormalTok{ domain.name,]}
\NormalTok{  domain.p }\OtherTok{\textless{}{-}}\NormalTok{ df[df}\SpecialCharTok{$}\NormalTok{opening\_name }\SpecialCharTok{==}\NormalTok{ domain.name,]}
  
\NormalTok{  n.d.s1 }\OtherTok{\textless{}{-}}\NormalTok{ openings.valid.str.sample[i, }\DecValTok{2}\NormalTok{]}
\NormalTok{  n.d.s2 }\OtherTok{\textless{}{-}}\NormalTok{ openings.valid.str.sample[i, }\DecValTok{3}\NormalTok{]}

\NormalTok{  N.d }\OtherTok{\textless{}{-}} \FunctionTok{nrow}\NormalTok{(domain.p)}
\NormalTok{  nh.d1 }\OtherTok{\textless{}{-}}\NormalTok{ domain.s1 }\SpecialCharTok{\%\textgreater{}\%} \FunctionTok{count}\NormalTok{(elo\_range, }\AttributeTok{.drop=}\ConstantTok{FALSE}\NormalTok{)}
\NormalTok{  nh.d2 }\OtherTok{\textless{}{-}}\NormalTok{ domain.s2 }\SpecialCharTok{\%\textgreater{}\%} \FunctionTok{count}\NormalTok{(elo\_range, }\AttributeTok{.drop=}\ConstantTok{FALSE}\NormalTok{)}
\NormalTok{  Nh.d }\OtherTok{\textless{}{-}}\NormalTok{ domain.p }\SpecialCharTok{\%\textgreater{}\%} \FunctionTok{count}\NormalTok{(elo\_range, }\AttributeTok{.drop=}\ConstantTok{FALSE}\NormalTok{)}
\NormalTok{  strata.size.prop }\OtherTok{\textless{}{-}}\NormalTok{ Nh.d[}\DecValTok{2}\NormalTok{]}\SpecialCharTok{/}\NormalTok{N.d}
  
  \CommentTok{\# Calculate white\textquotesingle{}s win proportion by each strata}
\NormalTok{  white.win.prop }\OtherTok{\textless{}{-}}\NormalTok{ domain.s1 }\SpecialCharTok{\%\textgreater{}\%}
    \FunctionTok{count}\NormalTok{(winner, }\AttributeTok{.drop=}\ConstantTok{FALSE}\NormalTok{) }\SpecialCharTok{\%\textgreater{}\%}
    \FunctionTok{group\_by}\NormalTok{(elo\_range) }\SpecialCharTok{\%\textgreater{}\%}
    \FunctionTok{mutate}\NormalTok{(}\AttributeTok{win.prop =}\NormalTok{ n }\SpecialCharTok{/} \FunctionTok{sum}\NormalTok{(n))}
  
  \CommentTok{\# The estimated aggregated win proportion for white}
\NormalTok{  white.prop }\OtherTok{\textless{}{-}}\NormalTok{ white.win.prop[white.win.prop}\SpecialCharTok{$}\NormalTok{winner }\SpecialCharTok{==} \StringTok{"white"}\NormalTok{, ] }
\NormalTok{  white.prop[}\FunctionTok{is.na}\NormalTok{(white.prop)] }\OtherTok{\textless{}{-}} \DecValTok{0}
\NormalTok{  white.p.str.est }\OtherTok{\textless{}{-}} \FunctionTok{sum}\NormalTok{(white.prop}\SpecialCharTok{$}\NormalTok{win.prop }\SpecialCharTok{*}\NormalTok{ strata.size.prop)}
  
  \CommentTok{\# The estimated aggregated variance of win proportion for white}
\NormalTok{  white.se.by.strata }\OtherTok{\textless{}{-}} \FunctionTok{bind\_cols}\NormalTok{(white.prop, }\AttributeTok{nh =}\NormalTok{ nh.d1}\SpecialCharTok{$}\NormalTok{n)}
\NormalTok{  white.se.by.strata }\OtherTok{\textless{}{-}}\NormalTok{ white.se.by.strata }\SpecialCharTok{\%\textgreater{}\%} \FunctionTok{mutate}\NormalTok{(}\AttributeTok{var.by.strata =}\NormalTok{ win.prop }\SpecialCharTok{*}\NormalTok{ (}\DecValTok{1}\SpecialCharTok{{-}}\NormalTok{win.prop)}\SpecialCharTok{/}\NormalTok{nh)}
\NormalTok{  white.se.by.strata }\OtherTok{\textless{}{-}} \FunctionTok{bind\_cols}\NormalTok{(white.se.by.strata, }\AttributeTok{strata.prop.sq =}\NormalTok{ strata.size.prop}\SpecialCharTok{$}\NormalTok{n}\SpecialCharTok{\^{}}\DecValTok{2}\NormalTok{)}
\NormalTok{  white.se.by.strata }\OtherTok{\textless{}{-}}\NormalTok{ white.se.by.strata }\SpecialCharTok{\%\textgreater{}\%} \FunctionTok{mutate}\NormalTok{(strata.prop.sq}\SpecialCharTok{*}\NormalTok{(}\DecValTok{1}\SpecialCharTok{{-}}\NormalTok{n}\SpecialCharTok{/}\NormalTok{nh)}\SpecialCharTok{*}\NormalTok{var.by.strata)}
\NormalTok{  white.se.by.strata[}\FunctionTok{is.na}\NormalTok{(white.se.by.strata)] }\OtherTok{\textless{}{-}} \DecValTok{0}
\NormalTok{  white.str.se }\OtherTok{\textless{}{-}} \FunctionTok{sqrt}\NormalTok{(}\FunctionTok{sum}\NormalTok{(white.se.by.strata}\SpecialCharTok{$}\StringTok{\textasciigrave{}}\AttributeTok{strata.prop.sq * (1 {-} n/nh) * var.by.strata}\StringTok{\textasciigrave{}}\NormalTok{))}
  
  \CommentTok{\# Calculate black\textquotesingle{}s win proportion by each strata}
\NormalTok{  black.win.prop }\OtherTok{\textless{}{-}}\NormalTok{ domain.s2 }\SpecialCharTok{\%\textgreater{}\%}
    \FunctionTok{count}\NormalTok{(winner, }\AttributeTok{.drop=}\ConstantTok{FALSE}\NormalTok{) }\SpecialCharTok{\%\textgreater{}\%}
    \FunctionTok{group\_by}\NormalTok{(elo\_range) }\SpecialCharTok{\%\textgreater{}\%}
    \FunctionTok{mutate}\NormalTok{(}\AttributeTok{win.prop =}\NormalTok{ n }\SpecialCharTok{/} \FunctionTok{sum}\NormalTok{(n))}
  
  \CommentTok{\# The estimated aggregated win proportion for white}
\NormalTok{  black.prop }\OtherTok{\textless{}{-}}\NormalTok{ black.win.prop[black.win.prop}\SpecialCharTok{$}\NormalTok{winner }\SpecialCharTok{==} \StringTok{"black"}\NormalTok{, ] }
\NormalTok{  black.prop[}\FunctionTok{is.na}\NormalTok{(black.prop)] }\OtherTok{\textless{}{-}} \DecValTok{0}
\NormalTok{  black.p.str.est }\OtherTok{\textless{}{-}} \FunctionTok{sum}\NormalTok{(black.prop}\SpecialCharTok{$}\NormalTok{win.prop }\SpecialCharTok{*}\NormalTok{ strata.size.prop)}
  
  \CommentTok{\# The estimated aggregated variance of win proportion for black}
\NormalTok{  black.se.by.strata }\OtherTok{\textless{}{-}} \FunctionTok{bind\_cols}\NormalTok{(black.prop, }\AttributeTok{nh =}\NormalTok{ nh.d2}\SpecialCharTok{$}\NormalTok{n)}
\NormalTok{  black.se.by.strata }\OtherTok{\textless{}{-}}\NormalTok{ black.se.by.strata }\SpecialCharTok{\%\textgreater{}\%} \FunctionTok{mutate}\NormalTok{(}\AttributeTok{var.by.strata =}\NormalTok{ win.prop }\SpecialCharTok{*}\NormalTok{ (}\DecValTok{1}\SpecialCharTok{{-}}\NormalTok{win.prop)}\SpecialCharTok{/}\NormalTok{nh)}
\NormalTok{  black.se.by.strata }\OtherTok{\textless{}{-}} \FunctionTok{bind\_cols}\NormalTok{(black.se.by.strata, }\AttributeTok{strata.prop.sq =}\NormalTok{ strata.size.prop}\SpecialCharTok{$}\NormalTok{n}\SpecialCharTok{\^{}}\DecValTok{2}\NormalTok{)}
\NormalTok{  black.se.by.strata }\OtherTok{\textless{}{-}}\NormalTok{ black.se.by.strata }\SpecialCharTok{\%\textgreater{}\%} \FunctionTok{mutate}\NormalTok{(strata.prop.sq}\SpecialCharTok{*}\NormalTok{(}\DecValTok{1}\SpecialCharTok{{-}}\NormalTok{n}\SpecialCharTok{/}\NormalTok{nh)}\SpecialCharTok{*}\NormalTok{var.by.strata)}
\NormalTok{  black.se.by.strata[}\FunctionTok{is.na}\NormalTok{(black.se.by.strata)] }\OtherTok{\textless{}{-}} \DecValTok{0}
\NormalTok{  black.str.se }\OtherTok{\textless{}{-}} \FunctionTok{sqrt}\NormalTok{(}\FunctionTok{sum}\NormalTok{(black.se.by.strata}\SpecialCharTok{$}\StringTok{\textasciigrave{}}\AttributeTok{strata.prop.sq * (1 {-} n/nh) * var.by.strata}\StringTok{\textasciigrave{}}\NormalTok{))}

  \CommentTok{\# Their difference}
\NormalTok{  estimates[i] }\OtherTok{\textless{}{-}}\NormalTok{ white.p.str.est }\SpecialCharTok{{-}}\NormalTok{ black.p.str.est}
  \CommentTok{\# Using pooled variance}
\NormalTok{  diff.se }\OtherTok{\textless{}{-}} \FunctionTok{sqrt}\NormalTok{(((n.d.s1}\DecValTok{{-}1}\NormalTok{)}\SpecialCharTok{*}\NormalTok{white.str.se}\SpecialCharTok{\^{}}\DecValTok{2} \SpecialCharTok{+}\NormalTok{ (n.d.s2}\DecValTok{{-}1}\NormalTok{)}\SpecialCharTok{*}\NormalTok{black.str.se}\SpecialCharTok{\^{}}\DecValTok{2}\NormalTok{)}\SpecialCharTok{/}\NormalTok{(n.d.s1}\SpecialCharTok{+}\NormalTok{n.d.s2}\DecValTok{{-}2}\NormalTok{))}
\NormalTok{  intervals[i,] }\OtherTok{\textless{}{-}}\NormalTok{ (white.p.str.est }\SpecialCharTok{{-}}\NormalTok{ black.p.str.est) }\SpecialCharTok{+} \FunctionTok{qt}\NormalTok{(}\FloatTok{0.975}\NormalTok{, n.d.s1}\SpecialCharTok{+}\NormalTok{n.d.s2}\SpecialCharTok{{-}} \DecValTok{2}\NormalTok{) }\SpecialCharTok{*}\NormalTok{ diff.se }\SpecialCharTok{*} \FunctionTok{c}\NormalTok{(}\SpecialCharTok{{-}}\DecValTok{1}\NormalTok{, }\DecValTok{1}\NormalTok{)}
\NormalTok{\}}

\NormalTok{openings }\OtherTok{\textless{}{-}} \FunctionTok{data.frame}\NormalTok{(openings.valid.str.sample}\SpecialCharTok{$}\NormalTok{name, intervals)}
\FunctionTok{colnames}\NormalTok{(openings) }\OtherTok{\textless{}{-}} \FunctionTok{c}\NormalTok{(}\StringTok{"name"}\NormalTok{, }\StringTok{"95.CI.lower"}\NormalTok{, }\StringTok{"95.CI.upper"}\NormalTok{)}
\NormalTok{white.higher }\OtherTok{\textless{}{-}}\NormalTok{ openings[openings}\SpecialCharTok{$}\StringTok{\textasciigrave{}}\AttributeTok{95.CI.lower}\StringTok{\textasciigrave{}} \SpecialCharTok{\textgreater{}} \DecValTok{0}\NormalTok{,]}
\NormalTok{white.lower }\OtherTok{\textless{}{-}}\NormalTok{ openings[openings}\SpecialCharTok{$}\StringTok{\textasciigrave{}}\AttributeTok{95.CI.upper}\StringTok{\textasciigrave{}} \SpecialCharTok{\textless{}} \DecValTok{0}\NormalTok{,]}
\NormalTok{openings}
\end{Highlighting}
\end{Shaded}

\begin{Shaded}
\begin{Highlighting}[]
\CommentTok{\# mean number of turns for white wins vs black wins?}
\NormalTok{white.win }\OtherTok{\textless{}{-}}\NormalTok{ srs.sample.s1[srs.sample.s1}\SpecialCharTok{$}\NormalTok{winner }\SpecialCharTok{==} \StringTok{"white"}\NormalTok{,]}
\NormalTok{black.win }\OtherTok{\textless{}{-}}\NormalTok{ srs.sample.s2[srs.sample.s2}\SpecialCharTok{$}\NormalTok{winner }\SpecialCharTok{==} \StringTok{"black"}\NormalTok{,]}

\NormalTok{white.win.turns.avg }\OtherTok{\textless{}{-}} \FunctionTok{mean}\NormalTok{(white.win}\SpecialCharTok{$}\NormalTok{turns)}
\NormalTok{black.win.turns.avg }\OtherTok{\textless{}{-}} \FunctionTok{mean}\NormalTok{(black.win}\SpecialCharTok{$}\NormalTok{turns)}

\NormalTok{n.w }\OtherTok{\textless{}{-}} \FunctionTok{nrow}\NormalTok{(white.win)}
\NormalTok{n.b }\OtherTok{\textless{}{-}} \FunctionTok{nrow}\NormalTok{(black.win)}

\NormalTok{srs.se }\OtherTok{\textless{}{-}} \FunctionTok{sqrt}\NormalTok{((}\DecValTok{1}\SpecialCharTok{{-}}\NormalTok{n.w}\SpecialCharTok{/}\NormalTok{N)}\SpecialCharTok{*}\FunctionTok{var}\NormalTok{(white.win}\SpecialCharTok{$}\NormalTok{turns)}\SpecialCharTok{/}\NormalTok{n.w }\SpecialCharTok{+}\NormalTok{ (}\DecValTok{1}\SpecialCharTok{{-}}\NormalTok{n.b}\SpecialCharTok{/}\NormalTok{N)}\SpecialCharTok{*}\FunctionTok{var}\NormalTok{(black.win}\SpecialCharTok{$}\NormalTok{turns)}\SpecialCharTok{/}\NormalTok{n.b)}
\NormalTok{(white.win.turns.avg }\SpecialCharTok{{-}}\NormalTok{ black.win.turns.avg) }\SpecialCharTok{+}\NormalTok{ z}\FloatTok{.95} \SpecialCharTok{*}\NormalTok{ srs.se }\SpecialCharTok{*} \FunctionTok{c}\NormalTok{(}\SpecialCharTok{{-}}\DecValTok{1}\NormalTok{, }\DecValTok{1}\NormalTok{)}
\end{Highlighting}
\end{Shaded}

\begin{Shaded}
\begin{Highlighting}[]
\NormalTok{white.win }\OtherTok{\textless{}{-}}\NormalTok{ white.str.sample[white.str.sample}\SpecialCharTok{$}\NormalTok{winner }\SpecialCharTok{==} \StringTok{"white"}\NormalTok{,]}
\NormalTok{black.win }\OtherTok{\textless{}{-}}\NormalTok{ black.str.sample[black.str.sample}\SpecialCharTok{$}\NormalTok{winner }\SpecialCharTok{==} \StringTok{"black"}\NormalTok{,]}

\CommentTok{\# Calculate average number of turns for white win by each strata}
\NormalTok{avg.turns.w }\OtherTok{\textless{}{-}}\NormalTok{ white.win }\SpecialCharTok{\%\textgreater{}\%}
  \FunctionTok{group\_by}\NormalTok{(elo\_range) }\SpecialCharTok{\%\textgreater{}\%}
  \FunctionTok{summarise}\NormalTok{(}\AttributeTok{mean\_turns =} \FunctionTok{mean}\NormalTok{(turns))}

\CommentTok{\# and for black win}
\NormalTok{avg.turns.b }\OtherTok{\textless{}{-}}\NormalTok{ black.win }\SpecialCharTok{\%\textgreater{}\%}
  \FunctionTok{group\_by}\NormalTok{(elo\_range) }\SpecialCharTok{\%\textgreater{}\%}
  \FunctionTok{summarise}\NormalTok{(}\AttributeTok{mean\_turns =} \FunctionTok{mean}\NormalTok{(turns))}

\CommentTok{\# The estimated average number of turns}
\NormalTok{white.avg }\OtherTok{\textless{}{-}} \FunctionTok{sum}\NormalTok{(Nh[}\DecValTok{2}\NormalTok{]}\SpecialCharTok{/}\NormalTok{N }\SpecialCharTok{*}\NormalTok{ avg.turns.w}\SpecialCharTok{$}\NormalTok{mean\_turns)}
\NormalTok{black.avg }\OtherTok{\textless{}{-}} \FunctionTok{sum}\NormalTok{(Nh[}\DecValTok{2}\NormalTok{]}\SpecialCharTok{/}\NormalTok{N }\SpecialCharTok{*}\NormalTok{ avg.turns.b}\SpecialCharTok{$}\NormalTok{mean\_turns)}

\CommentTok{\# The estimated se by strata}
\NormalTok{white.turn.strata  }\OtherTok{\textless{}{-}}\NormalTok{ white.win }\SpecialCharTok{\%\textgreater{}\%} \FunctionTok{group\_by}\NormalTok{(elo\_range) }\SpecialCharTok{\%\textgreater{}\%} \FunctionTok{summarise}\NormalTok{(}\AttributeTok{var =} \FunctionTok{var}\NormalTok{(turns))}
\NormalTok{white.turn.strata }\OtherTok{\textless{}{-}} \FunctionTok{bind\_cols}\NormalTok{(white.turn.strata, }\AttributeTok{nh =}\NormalTok{ nh.white}\SpecialCharTok{$}\NormalTok{n)}
\NormalTok{white.turn.strata }\OtherTok{\textless{}{-}} \FunctionTok{bind\_cols}\NormalTok{(white.turn.strata, }\AttributeTok{Nh =}\NormalTok{ Nh[}\DecValTok{2}\NormalTok{])}
\NormalTok{white.se.by.strata }\OtherTok{\textless{}{-}}\NormalTok{white.turn.strata }\SpecialCharTok{\%\textgreater{}\%} \FunctionTok{mutate}\NormalTok{(}\AttributeTok{var.by.str =}\NormalTok{  (}\DecValTok{1}\SpecialCharTok{{-}}\NormalTok{nh}\SpecialCharTok{/}\NormalTok{n)}\SpecialCharTok{*}\NormalTok{var}\SpecialCharTok{/}\NormalTok{nh)}
\NormalTok{white.se.by.strata }\OtherTok{\textless{}{-}} \FunctionTok{bind\_cols}\NormalTok{(white.se.by.strata, strata.size.prop)}
\NormalTok{white.se.by.strata }\OtherTok{\textless{}{-}} \FunctionTok{sqrt}\NormalTok{(}\FunctionTok{sum}\NormalTok{(white.se.by.strata}\SpecialCharTok{$}\NormalTok{n...}\DecValTok{6}\SpecialCharTok{\^{}}\DecValTok{2} \SpecialCharTok{*}\NormalTok{ white.se.by.strata}\SpecialCharTok{$}\NormalTok{var.by.str))}

\NormalTok{black.turn.strata  }\OtherTok{\textless{}{-}}\NormalTok{ black.win }\SpecialCharTok{\%\textgreater{}\%} \FunctionTok{group\_by}\NormalTok{(elo\_range) }\SpecialCharTok{\%\textgreater{}\%} \FunctionTok{summarise}\NormalTok{(}\AttributeTok{var =} \FunctionTok{var}\NormalTok{(turns))}
\NormalTok{black.turn.strata }\OtherTok{\textless{}{-}} \FunctionTok{bind\_cols}\NormalTok{(black.turn.strata, }\AttributeTok{nh =}\NormalTok{ nh.black}\SpecialCharTok{$}\NormalTok{n)}
\NormalTok{black.turn.strata }\OtherTok{\textless{}{-}} \FunctionTok{bind\_cols}\NormalTok{(black.turn.strata, }\AttributeTok{Nh =}\NormalTok{ Nh[}\DecValTok{2}\NormalTok{])}
\NormalTok{black.se.by.strata }\OtherTok{\textless{}{-}}\NormalTok{black.turn.strata }\SpecialCharTok{\%\textgreater{}\%} \FunctionTok{mutate}\NormalTok{(}\AttributeTok{var.by.str =}\NormalTok{  (}\DecValTok{1}\SpecialCharTok{{-}}\NormalTok{nh}\SpecialCharTok{/}\NormalTok{n)}\SpecialCharTok{*}\NormalTok{var}\SpecialCharTok{/}\NormalTok{nh)}
\NormalTok{black.se.by.strata }\OtherTok{\textless{}{-}} \FunctionTok{bind\_cols}\NormalTok{(black.se.by.strata, strata.size.prop)}
\NormalTok{black.se.by.strata }\OtherTok{\textless{}{-}} \FunctionTok{sqrt}\NormalTok{(}\FunctionTok{sum}\NormalTok{(black.se.by.strata}\SpecialCharTok{$}\NormalTok{n...}\DecValTok{6}\SpecialCharTok{\^{}}\DecValTok{2} \SpecialCharTok{*}\NormalTok{ black.se.by.strata}\SpecialCharTok{$}\NormalTok{var.by.str))}

\NormalTok{diff.se }\OtherTok{\textless{}{-}} \FunctionTok{sqrt}\NormalTok{(white.se.by.strata}\SpecialCharTok{\^{}}\DecValTok{2} \SpecialCharTok{+}\NormalTok{ black.se.by.strata}\SpecialCharTok{\^{}}\DecValTok{2}\NormalTok{)}
\NormalTok{(white.avg }\SpecialCharTok{{-}}\NormalTok{ black.avg) }\SpecialCharTok{+}\NormalTok{ z}\FloatTok{.95} \SpecialCharTok{*}\NormalTok{ diff.se }\SpecialCharTok{*} \FunctionTok{c}\NormalTok{(}\SpecialCharTok{{-}}\DecValTok{1}\NormalTok{, }\DecValTok{1}\NormalTok{)}
\end{Highlighting}
\end{Shaded}






\newpage
\singlespacing 
\end{document}
