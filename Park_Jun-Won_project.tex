\documentclass[11pt,]{article}
\usepackage[left=1in,top=1in,right=1in,bottom=1in]{geometry}
\newcommand*{\authorfont}{\fontfamily{phv}\selectfont}
\usepackage[]{mathpazo}


  \usepackage[T1]{fontenc}
  \usepackage[utf8]{inputenc}




\usepackage{abstract}
\renewcommand{\abstractname}{}    % clear the title
\renewcommand{\absnamepos}{empty} % originally center

\renewenvironment{abstract}
 {{%
    \setlength{\leftmargin}{0mm}
    \setlength{\rightmargin}{\leftmargin}%
  }%
  \relax}
 {\endlist}

\makeatletter
\def\@maketitle{%
  \newpage
%  \null
%  \vskip 2em%
%  \begin{center}%
  \let \footnote \thanks
    {\fontsize{18}{20}\selectfont\raggedright  \setlength{\parindent}{0pt} \@title \par}%
}
%\fi
\makeatother




\setcounter{secnumdepth}{0}




\title{White Advantage in Chess and How to Counter It  }



\author{\Large Jun Won (Lakon) Park (Group Leader)\vspace{0.05in} \newline\normalsize\emph{79453940, Group Leader, data collection, data analysis, report-writing}   \and \Large Sarah Li\vspace{0.05in} \newline\normalsize\emph{(your student number), data collection, data analysis, report-writing}  }


\date{}

\usepackage{titlesec}

\titleformat*{\section}{\normalsize\bfseries}
\titleformat*{\subsection}{\normalsize\itshape}
\titleformat*{\subsubsection}{\normalsize\itshape}
\titleformat*{\paragraph}{\normalsize\itshape}
\titleformat*{\subparagraph}{\normalsize\itshape}


\usepackage{natbib}
\bibliographystyle{plainnat}
\usepackage[strings]{underscore} % protect underscores in most circumstances



\newtheorem{hypothesis}{Hypothesis}
\usepackage{setspace}


% set default figure placement to htbp
\makeatletter
\def\fps@figure{htbp}
\makeatother


% move the hyperref stuff down here, after header-includes, to allow for - \usepackage{hyperref}

\makeatletter
\@ifpackageloaded{hyperref}{}{%
\ifxetex
  \PassOptionsToPackage{hyphens}{url}\usepackage[setpagesize=false, % page size defined by xetex
              unicode=false, % unicode breaks when used with xetex
              xetex]{hyperref}
\else
  \PassOptionsToPackage{hyphens}{url}\usepackage[draft,unicode=true]{hyperref}
\fi
}

\@ifpackageloaded{color}{
    \PassOptionsToPackage{usenames,dvipsnames}{color}
}{%
    \usepackage[usenames,dvipsnames]{color}
}
\makeatother
\hypersetup{breaklinks=true,
            bookmarks=true,
            pdfauthor={Jun Won (Lakon) Park (Group Leader) (79453940, Group Leader, data collection, data analysis, report-writing) and Sarah Li ((your student number), data collection, data analysis, report-writing)},
             pdfkeywords = {},  
            pdftitle={White Advantage in Chess and How to Counter It},
            colorlinks=true,
            citecolor=blue,
            urlcolor=blue,
            linkcolor=magenta,
            pdfborder={0 0 0}}
\urlstyle{same}  % don't use monospace font for urls

% Add an option for endnotes. -----


% add tightlist ----------
\providecommand{\tightlist}{%
\setlength{\itemsep}{0pt}\setlength{\parskip}{0pt}}

% add some other packages ----------

% \usepackage{multicol}
% This should regulate where figures float
% See: https://tex.stackexchange.com/questions/2275/keeping-tables-figures-close-to-where-they-are-mentioned
\usepackage[section]{placeins}


\begin{document}
	
% \pagenumbering{arabic}% resets `page` counter to 1 
%
% \maketitle

{% \usefont{T1}{pnc}{m}{n}
\setlength{\parindent}{0pt}
\thispagestyle{plain}
{\fontsize{18}{20}\selectfont\raggedright 
\maketitle  % title \par  

}

{
   \vskip 13.5pt\relax \normalsize\fontsize{11}{12} 
\textbf{\authorfont Jun Won (Lakon) Park (Group Leader)} \hskip 15pt \emph{\small 79453940, Group Leader, data collection, data analysis, report-writing}   \par \textbf{\authorfont Sarah Li} \hskip 15pt \emph{\small (your student number), data collection, data analysis, report-writing}   

}

}








\begin{abstract}

    \hbox{\vrule height .2pt width 39.14pc}

    \vskip 8.5pt % \small 

\noindent Research question: Is white at an advantage in chess and if so, what are
some optimal strategies for black to increase their winning probability?


    \hbox{\vrule height .2pt width 39.14pc}


\end{abstract}


\vskip -8.5pt


 % removetitleabstract

\noindent  

\hypertarget{introduction}{%
\subsubsection{Introduction}\label{introduction}}

For several centuries, millions of people worldwide have been playing
chess as a recreational and competitive board game at their homes, in
clubs, in tournaments, and even online nowadays. In the recent decades,
chess has been one of the most popular topic in machine learning and
artificial intelligence. The first move advantage has been researched
extensively since the end of 18th century, and many studies have been
shown that white has an inherent advantage.\\
\newline Although there are general set chess openings for black
according to white's first move, less research has been done on the
effects of those openings on the final outcome. This paper intends to
confirm white's first move advantage and study the relationship between
each opening or black and the final outcome.\\
\newline This paper's data collection consists basic player information
and game information of over 20000 chess games obtained exclusively from
Lichess, a very popular internet chess platform. The data includes game
length, number of turns, winner, player elo\(^*\), all moves in Standard
Chess Notation, Opening Eco\(^*\), Opening Name, and Opening Ply\(^*\)
(some stuff about sampling method and target population)\\
\newline Two research questions are explored via this paper: Is white at
an advantage in chess, and what are some optimal strategies for black to
increase their winning probability? (some summary about the result)\\
\newline ----------------------------------------------------------

\begin{flushleft}
Elo : A numerical measurement to quantify a player's skill level\newline
Eco : Standardised code for any given opening\newline
Ply : Number of moves in the opening phasenewline
\pagebreak
\end{flushleft}

\begin{center}\rule{0.5\linewidth}{0.5pt}\end{center}

\begin{flushleft}
\end{flushleft}

\hypertarget{analysis}{%
\subsubsection{Analysis}\label{analysis}}

\hypertarget{conclusion}{%
\subsubsection{Conclusion}\label{conclusion}}

\pagebreak
\begin{center}
\Large{References}
\end{center}





\newpage
\singlespacing 
\end{document}
